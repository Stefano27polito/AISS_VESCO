\section {Exercises}

The following exercises are designed to consolidate the understanding of the \textit{DID Document} data model and its key components, such as verification methods and service properties.  
Each task focuses on constructing and extending DID Documents according to the W3C DID specification (v1.1).  
Students are encouraged to reflect on the semantics of each property before reviewing the provided reference solution.

\subsection{Exercise 1: Two Verification Methods}
\textbf{Goal:} Create a simple DID Document containing two verification methods, without specifying any particular purpose for their usage.

\begin{verbatim}
{
  "@context": "https://www.w3.org/ns/did/v1.1",
  "id": "did:method_name:method_specific_id",
  "verificationMethod": [
    {
      "id": "did:method_name:method_specific_id#key-1",
      "type": "Multikey",
      "controller": "did:method_name:method_specific_id",
      "publicKeyMultibase": "… encoded key …"
    },
    {
      "id": "did:method_name:method_specific_id#key-2",
      "type": "JsonWebKey",
      "controller": "did:method_name:method_specific_id",
      "publicKeyJwk": {
        "kid": "key-34",
        "kty": "EC",
        "crv": "P-256",
        "alg": "ES256",
        "x": "… the X value …",
        "y": "… the Y value …"
      }
    }
  ]
}
\end{verbatim}

\noindent
\textbf{Comment:}  
The document defines two independent public keys. The first uses the \texttt{Multikey} encoding, while the second adopts a \texttt{JSON Web Key (JWK)} format.

\subsection{Exercise 2: Authentication and Assertion Methods}
\textbf{Goal:} Create a DID Document with two verification methods, each explicitly linked to a specific purpose: authentication and assertion.

\begin{verbatim}
{
  "@context": "https://www.w3.org/ns/did/v1.1",
  "id": "did:method_name:method_specific_id",
  "authentication": [
    {
      "id": "did:method_name:method_specific_id#key-1",
      "type": "Multikey",
      "controller": "did:method_name:method_specific_id",
      "publicKeyMultibase": "… encoded public key …"
    }
  ],
  "assertionMethod": [
    {
      "id": "did:method_name:method_specific_id#key-2",
      "type": "Multikey",
      "controller": "did:method_name:method_specific_id",
      "publicKeyMultibase": "… encoded public key …"
    }
  ]
}
\end{verbatim}

\noindent
\textbf{Comment:}  
Two separate public keys are published: one for authentication purposes and one for assertion.  
Both are controlled by the same DID subject.

\subsection{Exercise 3: Using Fragments and References}
\textbf{Goal:} Extend the previous document by defining references to the verification methods using fragments (\#key-1, \#key-2).

\begin{verbatim}
{
  "@context": "https://www.w3.org/ns/did/v1.1",
  "id": "did:method_name:method_specific_id",
  "verificationMethod": [
    {
      "id": "did:method_name:method_specific_id#key-1",
      "type": "Multikey",
      "controller": "did:method_name:method_specific_id",
      "publicKeyMultibase": "… encoded key …"
    },
    {
      "id": "did:method_name:method_specific_id#key-2",
      "type": "JsonWebKey",
      "controller": "did:method_name:method_specific_id",
      "publicKeyJwk": {
        "kid": "key-256",
        "kty": "EC",
        "crv": "P-256",
        "alg": "ES256",
        "x": "… the X value …",
        "y": "… the Y value …"
      }
    }
  ],
  "authentication": [
    "#key-1"
  ],
  "assertionMethod": [
    "#key-2"
  ]
}
\end{verbatim}

\noindent
\textbf{Comment:}  
Fragments are used to reference specific verification methods within the same DID Document.  
This structure promotes clarity and reuse of key definitions.

\subsection{Exercise 4: Authentication and Service Advertisement}
\textbf{Goal:} Create a DID Document that defines one verification method for authentication and one service endpoint for peer interaction.

\begin{verbatim}
{
  "@context": "https://www.w3.org/ns/did/v1.1",
  "id": "did:method_name:method_specific_id",
  "verificationMethod": [
    {
      "id": "did:method_name:method_specific_id#key-1",
      "type": "Multikey",
      "controller": "did:method_name:method_specific_id",
      "publicKeyMultibase": "… encoded key …"
    }
  ],
  "service": [
    {
      "id": "did:method_name:method_specific_id#message",
      "type": "Messaging",
      "serviceEndpoint": "https://www.polito.it/service/messaging"
    }
  ]
}
\end{verbatim}

\noindent
\textbf{Comment:}  
The document advertises a messaging service while defining a single verification method for authentication.  
Multiple services can be published as an array under the \texttt{service} property if needed.